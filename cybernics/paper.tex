% upLaTeX文書
\documentclass[uplatex,a4paper]{jsarticle}
% デフォルトのフォント設定のまま!
\usepackage{color}
\definecolor{myred}{rgb}{0.85,0,0.1}
\definecolor{mypink}{rgb}{1,0.92,0.92}
\setlength{\fboxsep}{2pt}


\title{サイバニクス 6月12日 授業レポート}
\author{201821636 村松直哉}
\date{\today}
\begin{document}
\maketitle
%
%
\section{安全}
\subsection{本質安全・機能安全}
本質安全は,機械が人間や環境に危害を及ぼす原因そのものを低減,あるいは除去することである.例として,モータにしようしている電源そのものを低出力にすることで,モータの出力を制限する.これにより,想定以上のモータ出力を制限する.
機能安全は,機能的な工夫を導入して,許容できるレベルの安全を確保することである.モータの例では,制御パラメータによって,出力の制限を与える.これは,電源自体の出力が低いわけではないので,パラメータを変化させる,ショートなどが起こると,想定以上の出力が発生する可能性がある.

\subsection{リスクアセスメント}
 リスクアセスメントは,作業場の潜在的な危険性または有害性を見つけ出し,これを除去,低減するため手法です.
 災害発生していない現場でも,潜在的な危険性や有害性が存在しており,これが放置させると,いつかは労働災害が発生する可能性がある.そのため,有害性を除去するリスクアセスメントが必要である.
 リスク低減方法には,「本質的な安全設計」「安全機能の実装」「情報の提供・運用」がある.

 本質的な安全設計の方策には,四つの柱がある.「危険除去設計」「フールプルーフ」「フェールセーフ」「冗長設計」である.
 危険除去設計とは,根本的に危険をのぞく設計である.鋭利な端部などをさけるなどがあげられる.
 フールプルーフは,人間はミスを犯すものだと前提をおき,使用者がまちがった使い方ができないようにする設計である.俗に言う「ポカよけ」などがそれにあたる.例としては,ドアを閉めなければ加熱できない電子レンジなどがある.
 フェールセーフは,故障がおきても危険を避けることができる設計である.つまり,壊れても安全状態に陥るようにすることである.例としては,電気ポットのコードに誤って触れても簡単に外れるなどがあげられる.
 冗長設計は,最低限必要な量より多めに装置を用意しておき,1つの装置が故障しても機能が失われない設計である.例として,Webサーバを2台用意し,片方のサーバが故障しても他方のサーバで対応するなどがある.

 安全機能の実装は,危険源による危害の程度を小さくすることが目的とされる.
 危険源の存在が,製品の機能と直結している場合,危険源自体を取り除くことは簡単で,はない.そこで,次に危険源による危害の程度を小さくすることを検討する.このステップで設計者が対応できることは非常にたくさんあり,可能な限りの対策を打つことが求めらる.例として,電気系統のヒューズやブレーカーなどがあげられる.

 情報の提供・運用には,危害の発生頻度を小さくするという目的がある.
 「付属文書(取扱い説明書)」「信号と警報装置」「表示」「標識」「警告文」などにより,使用上の情報を利用者に提示して,できる限り人間のミスを減らす.
 文書作成にあたっては「わかりやすく,やさしい言葉を使う」「あいまいな表現を使わない」 「図,表,スケッチ等を用い簡単に理解できるようにする」 ことが重要である.
 

\subsection{安全に関する各種規格}
 製品の安全規制を定める国際規格は,電気分野を専門に取り扱うIEC規格と,非電気分野を取り扱うISO規格から構成される.IEC規格を基準に安全性試験を実施し,その適合性を証明する国際的な制度をCB認証制度といい,これにより,各国の認証手続きの簡略化が可能になる.ISOには,品質マネジメントシステムのISO9001や,環境マネジメントシステムのISO14001などがある.
 本レポートでは,サイバニクス製品と関係の深いと考えた,「機械安全規格」について詳しく述べる.
 機械安全に関する規格の制定をするための基準として,IECとISOが共同作成したものが,ISO/IEC guide 51である.そこでは,国際規格をA, B, C規格と階層別に分類することを規定しており,最新の技術で製造される機械にも適用可能な規格構造が取られている.
 A規格(基本安全規格)は,あらゆる機械に適用できる基本概念や設計の原則などを規定した規格である.
 B規格(グループ安全規格)は,ある程度広範囲な機械群に対して適用できる安全性を規定した規格である.これは,さらにB1およびB2規格に分類できる.B1規格は,特定の安全的側面に関する規格(主に安全距離や騒音など)である.B2規格は,安全関連機器に関する規格(主にライトカーテンやインタロック装置など)である.
 C規格(個別製品安全規格)は,特定の機械又は機械群に対する詳細な安全性要求事項を規定する規格である.C規格が存在する機械には,C規格に従って設計することになる.C規格が存在しない機械の場合,A規格およびB規格に従って設計する.


\section{臨床}
\subsection{臨床安全}
 臨床安全とは,臨床試験を行い検討される安全性ことである.臨床試験は,ヒト(患者さんや健康な方)を対象として,薬や医療用具などの有効性や安全性などを検討するために行われる試験のことである.既存のものより有効であると期待される新しい治療法,診断法は.多くの患者さんの理解と協力を得て,「安全に実施できるのか」「期待どおりの効果を発揮するのか」を調べなければならない.
 臨床試験には,大きく分けて「治験」と「研究者(医師)主導臨床試験」がある.
 「治験」とは,厚生労働省に新たな医療機器としての承認を得ることを目的として行う臨床試験で,企業や医師が行う.治験の結果,厚生労働省から承認が得られれば,認められた病気の治療に対して,その医療機器を用いた治療ができるようになる.
 研究者(医師)主導臨床試験とは,研究者(医師)が主体となって非営利で行うもので,すでに承認された薬を組み合わせたり,手術や放射線治療を組み合わせるなどして,最良の治療法や診断法の確立などを目的としている.

\subsection{GCP}
 GCPとは,Good Clinical Practiceの略であり,「医薬品の臨床試験に関する基準」のことである.つまり,治験を実施する際に遵守すべき基準のことである.これは,薬事法第80条の2により,各治験実施関係者が治験を行う場合は,厚生労働省令で定める基準に従って実施しなければならないと規定されている.

\subsection{GMP}
 GMPとは,Good Manufacturing Practiceの略であり,製造業者(外国製造業者含む)および製造販売業者に求められる「適正製造規範」(製造管理・品質管理基準)のことであり,WHO(World Health Organization)等の国際機関や各国の規制当局が策定している.品質管理とは,医薬品等の原材料の入荷,検品,製造,製品の包装,出荷管理,製品保管,回収処理などに係る業務である.
 GMPは,以下の3つの要件を満たすことを目的としている.
人為的な誤りを最小限にする.
医薬品への汚染と品質変化を防止する.
高度な品質を保証する.

\subsection{IRB}
 IRBとは,Institutional Review Boardの略であり,「治験審査委員会」のことである.医学・歯学・薬学などの専門家およびそれ以外の者によって構成される医療機関の長,治験責任医師および治験依頼者から独立した委員会を指す.この委員会の責務は,特に,治験実施計画書(プロトコール)並びに被験者から文書によるインフォームド・コンセント(患者さんがその内容をよく理解した上で,本人の自らの考えにより,治験に参加することに同意するおこと)を得るのに使用される方法および資料などを審査,また継続審査を行うことによって,被験者の人権,安全および福祉の保護を確保することである.なお,主に欧州では独立倫理委員会/Independent Ethics Committee(IEC)が用いる.

\subsection{治験}
 人における試験を一般に「臨床試験」といい,医療薬の候補として,国の承認を得るための成績を集める臨床試験は,特に「治験」と呼ばれる.また,厚生労働省から「薬」として承認を受けるために行う臨床試験のことを「治験」と呼んでいる.治験は,GCPに従って行われる.



\section{法律}
\subsection{法的責任}
 法的責任とは,法律上,負わなくてはならない責任のことで,刑事責任と民事責任がある.
 特に医療機器の不具合を原因として医療事故が生じた場合の法的責任については,医療機器の欠陥に基づき医療事故が生じた場合,基本的には医療機器メーカーが無過失で賠償責任を負うことになる.しかし,例外的に,医療側にも無過失の賠償責任,ないし過失に基づく賠償責任が生じる場合がある.
 まず,医療機器に不具合があって患者に生命身体上の損害が発生し,それがこの機器の欠陥(製造物責任法第 2 条第 2 項)によるものであると認められた場合には,同法第 3 条に基づき,この医療機器の製造者に無過失で損害賠償責任が生じる.同条に基づく責任は無過失責任なので,機器を使用した医師や医療機関に過失がない場合,これらの者に責任は生じない.ただし,例外的に,医療機関に無過失で賠償責任が生じる場合があります.
 この機器を導入し設置している医療機関が国・公共団体の設置運営するものである場合には,この医療機器は,国家賠償法第 2 条の「公の営造物」に該当するため,その「設置・管理に瑕疵」がある限り,当該医療機関の設置管理者たる国・公共団体に同条に基づく無過失の損害賠償責任が生じる.この責任も無過失責任です.
~\cite{qi2016volumetric}

%
%

\bibliographystyle{ieee}
\bibliography{references}

\end{document}