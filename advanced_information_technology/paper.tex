% upLaTeX文書
% \documentclass[uplatex,a4paper]{jsarticle}
\documentclass[pdflatex,ja=standard]{bxjsarticle}
% デフォルトのフォント設定のまま!
\usepackage{color}
\usepackage{url}
\usepackage{graphicx}
\usepackage{algorithm}
\usepackage{algorithmic}

\definecolor{myred}{rgb}{0.85,0,0.1}
\definecolor{mypink}{rgb}{1,0.92,0.92}
\setlength{\fboxsep}{2pt}


\title{先端情報技術 10月3日 授業レポート}
\author{201821636 村松直哉}
\date{\today}
\begin{document}
\maketitle
%
%
\section{設問1}
\begin{description}
 \item[問] 学習サンプルさえ増やせば,音声認識の課題は全てDNNで解決すると考えられるか?それでも解決できない課題や問題があるとすれば,それは何か?
\end{description}

私はDNNでは音声認識に関する全ての問題は,解決できないと考える.
ここで「全ての問題の解決」とは,人間並みが高確率で成功させる課題に関して,同程度以上の確率で成功させることができること意味する.
主張の主な理由を2つ述べる.

\vspace{3mm}
1つはいわゆるone-shot learningの課題である.
DNNといわれるアルゴリズムは多くあるが,
その中でも,対象の学習データが1つのような状態を認識するのは非常に難しい.

新しい言葉や,対話者の造語などは,事前に大量の学習サンプルを確保することが難しい.
画像処理におけるone-shot learningの論文は,多く発表されている.
しかし,音声認識分野ではまだそれほどない.
また,one-shot learningの精度は,人間の認識精度よりも大きく劣っている.

この課題は,学習サンプル数の問題ではなく,特徴量抽出の問題である.
そのため,サンプルを増やしただけでは,解決できない課題の1つであると考える.

\vspace{3mm}
2つ目は,状況に応じた受け答えに関する課題である.
例えば「あれ」「これ」など,こそあど言葉を正確に認識することは難しい.
なぜなら,対話者同士の事前知識が必要である.
通常のDNNでは,記憶機能が明確になく,ニューロン同士の重みがその機能を果たしていると考えられる.
しかし,現在のニューラルネットワークでは,具体的な情報を引き出すことができない.
記憶を参照するような対話を認識するためには,DNNが具体的なデータを扱えるようにする必要がある.


以上のような理由から,現在のDNN技術だけで,学習サンプルさえ増やせば,音声認識課題を全て解決することができないと考える.


\end{document}
